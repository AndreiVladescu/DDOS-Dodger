\chapter*{Abstract}
\addcontentsline{toc}{chapter}{Abstract}

\textbf{Abstract} \\[0.5em]

This work evaluates the effectiveness of Moving Target Defense (MTD) strategies in bolstering the resilience of Internet of Things (IoT) networks against Distributed Denial of Service (DDoS) botnet attacks. With the rapid adoption of these diverse ecosystems of resource-constrained IoT devices, security gaps have widened significantly, increasing the number of exploitable attack vectors, adding vulnerabilities in said networks. In this thesis we aim to evaluate the application of MTD mechanisms along Software-Defined Networking (SDN) techniques to shift network parameters dynamically, with the aim of disrupting an adversary's ability to perform DDoS attacks.

Inherent to MTD, this approach isn't designed to block an attack outright, but rather make it costlier for the malign actor to sustain it. Through simulated experiments into a public WAN network, we demonstrate that this strategy makes coordinated DDoS campaigns against public IoT infrastructure economically unsustainable, ultimately enhancing the security posture. 


\vspace{1.5em}
\textbf{Rezumat} \\[0.5em]

Această lucrare evaluează eficiența strategiilor de tip Moving Target Defense (MTD) în consolidarea rezilienței rețelelor de dispozitive Internet of Things (IoT) împotriva atacurilor de tip Distributed Denial of Service (DDoS) lansate de botnet-uri. Odată cu adoptarea accelerată a acestor ecosisteme diverse de dispozitive IoT cu resurse limitate, decalajele de securitate s-au extins semnificativ, crescând numărul vectorilor de atac exploatabili și adăugând vulnerabilități suplimentare în cadrul rețelelor.

În această teză ne propunem să evaluăm aplicarea mecanismelor MTD în combinație cu tehnici de tip Software-Defined Networking (SDN), pentru a modifica dinamic parametrii rețelei, având ca scop perturbarea capacității unui adversar de a desfășura atacuri DDoS.

Prin natura sa, abordarea MTD nu este concepută pentru a bloca direct un atac, ci pentru a-l face mai costisitor și mai greu de susținut pentru actorul malițios. Prin experimente simulate într-o rețea WAN publică, demonstrăm că această strategie face ca desfășurarea unor campanii DDoS coordonate împotriva infrastructurii IoT publice să devină nesustenabilă din punct de vedere economic, îmbunătățind astfel postura generală de securitate.