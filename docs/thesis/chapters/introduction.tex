\chapter{Introduction}
\label{chap:intro}
\section{Motivation and Context}
In the recent years, Internet of Things (IoT) devices have transformed both consumer and industrial landscapes by  enabling sensors, actuators, appliances, and other smart devices to access the Internet. Today’s IoT deployments range from simple home‐automation hubs controlling lights or thermostats to large‐scale industrial installations or critical‐infrastructure monitoring, such as electrical grid sensors. As of 2024, it is estimated that over 18.8 billion IoT endpoints are deployed worldwide, a number projected to exceed 28 billion by 2027\cite{iotanalytics2024}. Despite this explosive growth, the majority of these devices remain resource‐constrained: they often run on low‐power microcontrollers, possess limited memory and compute capacity, and lack rigorous security controls\cite{sh_selvaraj2023}. The result is a highly heterogeneous ecosystem in which an adversary can easily compromise a subset of devices and leverage them for larger‐scale attacks\cite{sh_xenofontos2022}.

Studies regarding the attack surface and cybersecurity incidents statistics\cite{iot_vulnerable1}\cite{iot_vulnerable2} show that malign actors favor these devices to penetrate the security of a network, then pivoting to attack the other segments of the network\cite{iot_vulnerable2}. Also, IoT devices pose a danger by being added to botnets, such as the infamous Mirai botnet\cite{iot_vulnerable1}, whose code was made open-source and is now used by other botnets.

A particularly dangerous attack is the Distributed Denial of Service (DDoS) attack. Traditionally, DDoS campaigns have targeted servers, data centers and cloud infrastructure, consuming available bandwidth or exhausting system resources to render services unavailable. In the context of IoT, however, the attacker’s cost for conducting such an attack is reduced significantly, since the device is not going to handle a lot of traffic. As previously  stated, a botnet of other IoT devices, such as cameras, routers, or smart thermostats can be enrolled to generate massive volumes of traffic, thereby denying access to the device, or even breaking it. Worse still, the target may itself be an IoT service—such as a building‐management system or a remote health‐monitoring server—where even a brief outage can have serious real‐world consequences. For resource‐constrained IoT endpoints, conventional DDoS mitigation techniques (access control rules, firewall rules or signature‐based intrusion detection) may not be viable: most off‐the‐shelf IoT gateways cannot handle high‐speed packet inspection, and embedding sophisticated detection logic into every sensor node is simply not feasible.

Against such attack, Moving Target Defense (MTD) has emerged as a promisign mechanism \cite{mtd_vol1}\cite{jafarian2014}\cite{macfarland2015} to increase the attacker's workload and reduce the risk of succesful compromise. MTD seeks to increase the attack surface by introducing uncertainty for the adversary, in any way, shape or form imaginable, ranging from IP changes to VM shuffling. These types of shuffles also foil the attacker's recconaissance and defeat the cyber kill-chain's\cite{lockheed_killchain} later phases. When combined with Software-Defined Networking (SDN), it becomes possible to orchestrate the changes at the network level centrally, through rules. Whereas SDN is already used in cloud environmentes, it wasn't yet tested in public infrastructure in the context of securing IoT networks against DDoS attacks.

\section{Problem Statement}
This thesis aims to address the following problems and find a common solution: how can we design and evaluate a relatively lightweight architrecture, that will have an impact against DDoS botnets? Moreover, given the current landscape of IoT ecosystems, is there a possible solution to also make it usable in the public infrastructure, while using IPv4?

To describe a simplistic environment, in today's IoT deployments, a significant challenge arises from the fact that not many devices are available with an IPv6 networking stack, thus being dependent on IPv4, often with NAT gateways. This is an inherent problem of the IPv4, since we cannot use as many IPs without exhausting the address space. This isn't a problem with IPv6, as we can freely have global unique address for each device.
This limitation is present when, for example, the system is deployed in the public infrastructure, with an IP provided by the ISP's DHCP server. The IPs are most of the time shuffled when the lease time expires and the router borrows another one. The idea described in this work is that we can use the DHCP's random address leasing as a free IP changing mechanism.

To elaborate more, given that conventional DDoS attacks imply volumetric attacks to bring down services, we can "dodge" these attacks by requesting new IPs from the ISP's DHCP, so that a pool of IPs is not required at all times to perform this switch.
        
\section{Research Contributions}

We aim to contribute to the larger scientific community by combining MTD and SDN in a specialized environment, where a test microcontroller is the ultimate target of an artificial DDoS botnet, using realistic testing methods. The proposed architecture uses a device-agnostic approach, where the end-user may be free to add other DDoS detection methods than the ones used in this work. Also, the end-user is enabled to use as many layers of indirection to defeat other types of attacks, not tested in this thesis.

We built a simulation environment comprised of Virtual Machines (VMs) and Docker containers, as well as physical devices, to have a realistic testbed. In this testbed we not only measured if the system is protected from volumetric TCP/UDP flood attacks, but also measured key parameters, including failure rates, power consumption and latency.

The SDN architecture is implemented with the help of Python scripts, and the Northbound/Southbound APIs are implemented using MQTT for ease-of-use and later modifications.

\section{Thesis Outline}
The remainder of this thesis is organized as follows: 
%Chapter II introduces the context and existing works in the literature and also initiates the reader into the technical and theoretical key concepts. In Chapter III, we present the algorithms and theoretical framework which form the base of our system. Following this, the performance of the solution is tested and evaluated in the next section, in Chapter IV. Chapter V is reserved for conclusions about the system's performance, along with highlights of key findings.

In chapter II, named "Background and Related Work", theoretical key concepts are introduced to the reader, as a technical foundation for the rest of the thesis. This is taking into account how MTD works as a paradigm, along with some already-used MTD mechanisms, and how SDN is structured. Finally, we survey existing works done in the direction of MTD, in partiucular network-level defenses, and the type of surface dynamic (IP or port hopping) and scope (host-level, network-level).

In chapter III, "System Architecture", we begin by formalizing our threat model and defining security and performance assumptions. We then present detailed design constraints specific to IoT networks (e.g. physical connections and data transfer medium). Next, we describe the proposed SDN-driven MTD framework, including block diagrams, algorithms, and pseudocode detailing how address shuffling and flow-rule updates are orchestrated. Implementation details, such as controller logic, switch configurations, and the mechanism for client notifications, are also covered.

After making defining how the system works, in the fourth chapter, "Testing and Evaluation", we define the metrics used to evaluate the solution, including latency, failure rates and power consumption. We describe our experimental environment and outline specific test scenarios. We present statistical results in the form of graphs, then analyze the trade-offs between security gains and performance overhead. We also discuss a few edge cases, such as controller failure and high-churn IoT populations.

The final chapter is reserved for conclusions. We are summarizing how the system performed in accordance with our initial goals. We discuss the main findings and limitations of our current prototype. Finally, we propose ways this work can be improved, such as adding more intelligent DDoS detection systems and how to better test the system in a more realistic scenario.